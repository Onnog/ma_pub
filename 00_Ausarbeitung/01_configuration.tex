%!TEX encoding = UTF-8 Unicode
% =====================================================
% Seite einrichten
% =====================================================
%\usepackage[left=2cm,right=2cm,top=2cm,bottom=2cm,includeheadfoot]{geometry}

% =====================================================
% Schrift
% =====================================================

%\usepackage[it]{mathpazo}
%\linespread{1.05}

%\usepackage{lmodern}
%\renewcommand*\familydefault{\sfdefault} %% Only if the base font of the document is to be sans serif

%\usepackage{antiqua}

% =====================================================
% sonstige Einstellungen
% =====================================================
%\flushbottom % Textfluss schoen unten ausrichten
%\footnotesep12pt % Abstand Text / Fussnote
%\setlength{\parindent}{0pt} \setlength{\abovecaptionskip}{-6pt}
%\setlength{\belowcaptionskip}{0pt} \setlength{\intextsep}{18pt}
%\renewcommand{\baselinestretch}{1.2} % Zeilenabstand
%\setcounter{secnumdepth}{4}
%\newcommand{\p}[1]{\texttt{#1}}
%\nonfrenchspacing

% --- Abkürzungsverzeichnis EInstellen------
% START % Näheres siehe http://my.opera.com/timomeinen/blog/show.dml/68644
%\usepackage{nomencl}
% Andere Überschrift
%\renewcommand{\nomname}{List of Abbreviations}
% Punkte zw. Abkürzung und Erklärung
%\setlength{\nomlabelwidth}{\hsize}
%\renewcommand{\nomlabel}[1]{#1 \dotfill}
% Zeilenabstände verkleinern
%\setlength{\nomitemsep}{-\parsep}
%\makenomenclature
%--------------------------------------------------------

% =====================================================
% Kopf- und Fusszeile formatieren
% =====================================================
%}